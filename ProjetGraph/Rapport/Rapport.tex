\input{header.tex}
\title{Rapport - Projet Individuel \no 20\\\Large Test de la connexité d'un graphe}
\fancyhf{}
\lhead{\leftmark}
\lfoot{COLEAU Victor \& COUCHOUD Thomas}
\rfoot{Page~\thepage}

\begin{document}
	\maketitle
	\tableofcontents
	\chapter{Présentation du sujet}
		L'objectif de ce projet est de réaliser l'implémentation d'une fonction permettant de déterminer si un graphe est connexe tout en se basant sur la librairie que nous avons développé précédemment lors du projet \no 2.		
		
		Nous nous intéressons plus particulièrement aux graphs non orientés mais il est a noter que la méthode implémentée est aussi capable de s'appliquer sur des graphs orientés.
		
		La connexité d'un graphe se détermine selon les chemins réalisables. En effet si pour toute paire de sommets du graphe il existe un chemin entre eux, alors le graphe est défini comme connexe. De manière plus visuelle, un graphe connexe est un graphe don't tout les sommets font partis du même "bloc" (il n'y a pas plusieurs regroupements de sommets, mais uniquement un).
	
	\chapter{Architecture}
		%TODO Victor: CGraphToolbox
		
		%TODO Victor: CGraph - GRAgetRechableIndices
		
		%TODO Victor: CVertex - VERgetRechableIndices
		
		\img{../../classDiagram.png}{Diagramme de classes}{0.34}{0} %TODO Thomas: Update UML
		
	\chapter{Choix de codage}
		%TODO Victor: Choix de recopie de graph dans la toolbox
		
		%TODO Victor: Choix du tableau avec a l'indice 0 le nombre d'elements
		
		%TODO Victor: Choix de separer la methode pour transformer un graph oriente en un graph non oriente de la methode pour verifier la connexite
		
	\chapter{Tests effectués}
		Afin de pouvoir valider le fonctionnement de notre code, nous avons réalisé quelques tests. Certains ont été réalisés manuellement mais un bon nombre d'entre eux ont été écrits sous forme de "test unitaires". Ceux-ci sont présents dans les fichiers CXXXTest.cpp qui vont respectivement tester leur classe XXX. Une grosse partie de ceux-ci proviennent des tests effectuée lors de la partie 2 du projet. Seulement la classe CGraphToolboxUnit a été ajoutée.
		
		Celle-ci contient les tests suivants
		\begin{itemize}
			\item Test de la transformation d'un graph orienté en un graph non orienté.
			\item Test de la fonction indiquant si un chemin existe entre deux sommets.
			\item Test de la fonction indiquant si un chemin existe entre deux sommets avec pour sommet de départ un sommet n'existant pas.
			\item Test de la fonction indiquant si un chemin existe entre deux sommets avec pour sommet d'arrivée un sommet n'existant pas.
			\item Test de la fonction indiquant la connexité d'un graph.
		\end{itemize}
		
		De plus un test valgrind est réalisé sur la phase d'executions des tests.
			
	\chapter{Conseils d'utilisation}
	Le programme a été conçu et compilé de sorte que l'exécutable puisse prendre en arguments un fichier source de graph formaté comme défini dans le sujet.
	
	Suite à cela, il va exécuter les instructions suivantes :
	\begin{itemize}
		\item Affichage de la connexité du graph.
		\item Affichage de la connexité du graph après transformation en un graph non orienté. \\
	\end{itemize}
	Pour une utilisation en tant que librairie, tout se fait à partir d'un objet CGraphToolbox. Celle-ci dupliquera le graph passé en paramètre lors de sa création et offrira différentes fonctionnalités sur ce dernier. Cela comprend:
	\begin{itemize}
		\item Transformation du graph en un graph non orienté.
		\item Savoir si un chemin existe entre deux sommets.
		\item Savoir si le graph est connexe.\\
	\end{itemize}
	
	Pour toute information supplémentaire sur les méthodes, se référer aux cartouches d'entête présents dans les fichiers .h.
\end{document}
